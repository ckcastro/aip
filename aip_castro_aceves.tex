\documentclass{aip-cp}

\usepackage[numbers]{natbib}
\usepackage{rotating}
\usepackage{graphicx} 
\usepackage{mathtools}

%------------------------------------------------------------
% Frequently used long commands
\newcommand{\smu}{Southern Methodist University}
\newcommand{\pt}{$\mathcal{PT}$}
\newcommand{\MM}{{\mathcal M}}

%-------------------------------------------------------------


% Document starts
\begin{document}

% Title portion
%The Title Goes Here with Each Initial Letter Capitalized
\title{Optical mode stability and dynamics in Nonlinear Twisted PT-symmetric Structures }

\author[aff1]{Claudia Castro-Castro\corref{cor1}}
%\eaddress[url]{http://www.aip.org}
\author[aff1]{Alejandro B. Aceves}
%\eaddress{anotherauthor@thisaddress.yyy}

\affil[aff1]{Department of Mathematics, \smu, Dallas, Texas 75275, USA. }
\corresp[cor1]{Corresponding author: ccastrocastr@smu.edu.}

\maketitle

\begin{abstract}
Parity Time (\pt-)symmetry and symmetry breaking dynamics is being considered in an optical setting consisting of nonlinear multi-core fibers, where the axis of these cores follow a helical path.  
\end{abstract}

% Head 1
\section{INTRODUCTION}

\pt-symmetry is a property first considered in a Quantum Mechanics (QM), where a the potential in Schr\"odinger's equation in complex with the real part being even and the imaginary part being odd. One can show that a this turns out to be a necessary condition for the spectrum (Energy) to be real. Not surprisingly, as it has been the case with other fundamental phenomena considered in a QM setting (Anderson localization being one that stands out) or a nonlinear waves setting (Ocean rogue waves), \pt-symmetry can be studied and even tested experimentally in an optics framework.  

Here we present theoretical and numerical work on the stability properties of linght propagating in multi-core fiber arrays where the axis of each core follows a helical path (see figure 1. CLAUDIA: Add the figure of the twisted core fiber array). Adding twist to this optical systems induces a phase factor in the nearest neighbor coupling constants, thus a new degree of freedom to a system that has nonlinearity and loss/gain strength representing a discrete \pt property. As one can intuitively notice, twist is a topological property, so this optical configuration was recently considered~\cite{cleo} to show that Aharonov-Bohm-like suppression of optical tunneling in twisted multicore fibers can persist under highly nonlinear conditions. 

In a recent work of Longhi~\cite{longhi2016pt} only considers the linear case where he shows that the
transition from unbroken to broken ~\pt~ phases can be conveniently controlled by a suitable geometric 
twist of the fiber. The fiber twist which is responsible for the introduction of additional phase 
terms in the propagating fiber modes $e^{\pm i\phi}$, $\phi$ is defined as the Peierls phase, 
\ $\phi=4\pi^{2}\varepsilon n_{s}r_{0}^{2}/N\lambda$, where $n_{s}$ is the substrate (cladding) 
refractive index, $r_{0}\simeq R_{0}$ is an effective ring radius, $\lambda$
is the wavelength of the propagating field \cite{longhi2007light}. 

It is the aim of this preliminary work to explore \pt for larger arrays and to account for nonlinear effects~\cite{zhang}.
For that, we explore how the dynamics of nonlinear fiber with six cores is affected by an
induced twist in the scenario when gain/loss is not present in the model and when
there is alternating gain/loss profile.

The model for light propagating in a multi-core array read 
\begin{equation}
i\frac{dc_n}{dz} = k( e^{-i\phi} c_{n+1} + e^{i\phi} c_{n-1} ) +  i\gamma_n c_n + \sigma|c_n|^2c_n.
\label{eq:nonlinear-multicore}
\end{equation}
Here $c_n$ represent the complex amplitudes of a light beam traveling along the direction 
of propagation $z$, $\gamma_n$ defines the optical gain $(\gamma_n > 0)$ or loss $(\gamma_n < 0)$ rate at 
site $n$, $\sigma$ is the strength of the Kerr-nonlinearity, $n = 1,2,\dots 6 $ with periodic boundary conditions  $c_{ n+N } = c_n$,
for the initial input profile at core 1 (off-gain/loss) i.e. $c_1(0) = 1$, $c_2(0) = 0 $, \dots, $c_6(0) = 0$.

Possible stationary solutions take the form 
\[ c_{n} = \begin{cases*}
                    A_{n}e^{-i\lambda z} & if  $n=1,3,5,$ \\
                    B_{n}e^{-i\lambda z} & if  $n=2,4,6.$
           \end{cases*} \]%
\noindent
where $A_n, B_n$ who do not depend on $z$, satisfy a nonlinear system of algebraic equations,
\begin{eqnarray}
-\lambda A_{n}	+	k\left(e^{-i\phi}B_{n+1}+e^{i\phi}B_{n-1}\right)+i\gamma_{n}A_{n} + \sigma|A_n|^2A_n=0, \label{eq:stationary_sol_reducida1}\\
-\lambda B_{n}	+	k\left(e^{-i\phi}A_{n+1}+e^{i\phi}A_{n-1}\right)+i\gamma_{n}B_{n} + \sigma|B_n|^2B_n=0. \label{eq:stationary_sol_reducida2}
\end{eqnarray}
\noindent
Furthermore, if we assume a Continuous-Wave (CW) type of mode $A_n=A$,  $B_n = B$, and  $\gamma = 0 $, we can find the solutions using 
Eq. (\ref{eq:stationary_sol_reducida1})-(\ref{eq:stationary_sol_reducida2}) 
by means of fixed-point Newton iterations. For a particular initial guess of the root finding process, 
we portray the solutions found as functions of $\lambda$ in Fig. (\ref{fig:stationary-0}) for $\gamma = 0$. In Fig. (\ref{fig:stationary-0})
we can, for instance, see nontrivial solutions when $\lambda > \sqrt{3}$ when $\phi=\pi/6.$

% Figure
\begin{figure}[h]
  \centerline{\includegraphics[width =.5\textwidth]{./figures/stationary-sols-order0-gainloss-0-twisted}}
  \caption{(Color online) $\Re(A)$, $\Re(B)$, $\Im(A)$, and $\Im(B)$ numerical solutions of Eq. (\ref{eq:stationary_sol_reducida1})-(\ref{eq:stationary_sol_reducida2}), 
  $k=1$,  $\sigma = 1$, $\gamma =0$, several values of $\phi$. The color code is blue for $\pi/8$, orange for $\pi/6$, yellow for $\pi/5$, and purple for $\pi/4$.}
\label{fig:stationary-0}
\end{figure}

To consider the case where the gain-loss strength is close to zero, we can take a perturbative approach and expand both the coefficients and the solutions in terms of a 
small parameter   $\epsilon \ll 1$:
$\gamma^{(0)} + \epsilon \gamma^{(1)}+\dots$. For the search of the nonlinear modes in terms of this small parameter, we take a continuation approach to search for solutions in a neighborhood of $((A^{(0)},B^{(0)})^T,\gamma^{(0)})$, here $A^{(0)}$ denotes the solution 
associated to the case $\gamma = \gamma^{(0)} = 0$. By following the expansion in terms of $\epsilon$ we find that 
\begin{center}
$\begin{array}{rl}
\gamma =& \gamma^{(0)} + \epsilon \gamma^{(1)} + \epsilon^2 \gamma^{(2)}+\dots,\\
\lambda =& \lambda^{(0)} + \epsilon \lambda^{(1)} + \epsilon^2 \lambda^{(2)}+\dots,\\
A=&A^{(0)}+\epsilon A^{(1)}+\epsilon^2 A^{(2)}+\dots, \textrm{and}  \\
B=&B^{(0)}+\epsilon B^{(1)}+\epsilon^2 B^{(2)}+\dots
\end{array}$
\end{center}

Summarizing, the terms proportional to $\epsilon$ give equations for $A^{(1)}$ and $B^{(1)}$ of the form


\begin{equation}
\mathcal{M}^{(0)}U^{(1)}=\Psi^{(0)} U^{(0)}, 
\label{eq:stationary_sol_order1_matrix1}
\end{equation}

Define matrices\\


One finds that $\MM^{(0)T}$ will have a nontrivial nullspace whenever 

\begin{equation}
 4k^2cos^2\phi = 4\sigma^2x_1^{(0)2}y_1^{(0)2}.
 \label{eq:nontrivial_nullspace}
\end{equation}

As way of an example, we present results for $\sigma=k=1$, which are summarized  in Fig. (\ref{fig:rank_condition}), where the left and right-hand sides of Eq. 
(\ref{eq:nontrivial_nullspace}) are shown as dotted and solid curves respectively for several values of the phase $\phi$. We observe that for every value of $\phi$ tested,
there is a $\lambda^{(0)}$ where Eq. (\ref{fig:rank_condition}) is satisfied. In such case $\MM^{(0)}$ will not have an inverse in the ordinary sense. Then we can state the condition for existence of a solution from above, as follows
\begin{equation}
\Psi^{(0)} U^{(0)}\bot \mathrm{Null}\mathcal{M}^{(0)T}.
 \label{eq:solvability-condition}
\end{equation}

\begin{figure}
 \centering
  \includegraphics[width =.4\textwidth]{./figures/FullRankCondition.pdf}
  \caption{Full Rank Condition for Several values of $\phi$. The color code is blue for $\pi/8$, orange for $\pi/6$, yellow for $\pi/5$, and purple for $\pi/4$.}
\label{fig:rank_condition}
\end{figure}

From the description of  $\mathrm{Null}\MM^{(0)T}$ as

\begin{equation}
\mathrm{Null}\MM^{(0)T}=\left\{ \left(-\frac{m_{13}}{m_{12}} \alpha, -\frac{m_{13}}{m_{12}}\beta, \beta, \alpha \right)^{T}\arrowvert \alpha,\beta \in \mathbb{R} \right\}  
\end{equation}

we end up with the solvability condition
\begin{equation}
 \alpha\left[\left(\gamma^{(1)}-\lambda^{(1)}\right)\frac{{\sigma x_{1}^{(0)2}y_{1}^{(0)}}}{k\cos{\phi}}+\left(\gamma^{(1)}+\lambda^{(1)}\right)y_2^{(0)}\right]+\beta\left[\left(-\gamma^{(1)}-\lambda^{(1)}\right)\frac{{\sigma x_{2}^{(0)2}y_{2}^{(0)}}}{k\cos{\phi}}+\left(-\gamma^{(1)}+\lambda^{(1)}\right)y_1^{(0)}\right]=0.
\label{solvability_condition}
\end{equation}

The minimum-norm solution to a least squares  problem with coefficient matrix $\MM^{(0)}$
can be expressed in terms of the pseudoinverse $\mathcal{M}^{(0)+}$ as $U^{(1)}=\mathcal{M}^{(0)+}\Psi^{(0)} U^{(0)}$.

COMMENT: Claudia, you need to find a smooth way to finish this section

\section{Stability analysis}
Once we find solutions following the approach just described above, a natural question is to determine stability properties. For that we introduce small perturbations $\delta_{a}\left(z\right)$ and $\delta_{b}\left(z\right)$
\begin{center}
$A_n=\left(A^{s}+\delta_{a}\left(z\right)\right)e^{-i\lambda z}$ and $B_n=\left(B^{s}+\delta_{b}\left(z\right)\right)e^{-i\lambda z}$
\end{center}

Perturbations the satisfy the linearized equations
\begin{eqnarray}
i\frac{d\delta_a}{dz}=& -\lambda \delta_a +2 k \cos{\phi}\delta_b+i\gamma\delta_a+2\sigma|A^s|^2\delta_a+ \sigma(A^s)^2\delta_a^*, \label{eq:perturbation_linear_a}\\
i\frac{d\delta_b}{dz}=& -\lambda \delta_b +2 k \cos{\phi}\delta_a-i\gamma\delta_b+2\sigma|B^s|^2\delta_b+ \sigma(B^s)^2\delta_b^*, \label{eq:perturbation_linear_b}
\end{eqnarray}


Exploring the spectrum of Equation (\ref{eq:perturbation_linear_a}-\ref{eq:perturbation_linear_b}), we were able 
to numerically find the leading and first correction terms for the real and imaginary parts of the stationary 
solutions $A$ and $B$. Here we only consider 
one branch of the solution for $k = 1$,  with $\lambda$ and $\sigma$ continuously increasing up to the value 5, and varying the twist rate $\phi$ and in this regime,  
we examine the instability growth $p$ defined as the real part 
of the eigenvalue with the largest positive real part. 

COMMENT: Again find a smooth way to conclude this section 

\begin{figure}[h!]
    \centering
\includegraphics[width=0.45\textwidth]{./figures/instability-propagation-phi1}  \label{fig:1pt}
\includegraphics[width=0.45\textwidth]{./figures/instability-propagation-phi39} \label{fig:2pt}
\caption{(Color online) Instability propagation $p$ as function of the propagation constant $\lambda$ continuously increasing 
  up to $\lambda = 5$ for $k = 1$, $\epsilon = 0.05$, and $\sigma$ spanning from $(0.1,5)$, and $\phi =0$ (left) and $\phi =1.4212$ (right)}
\label{fig:linear-alternating-pt}
\end{figure}


\section{Summary}
We have presented initial studies on the linear stability properties of numerically constructed stationary solutions for a \pt-symmetry nonlinea twisted fiber array with 
gain-loss strength close to zero value. Our first observations suggest that the addition of twist will generically destabilize the system  for small amplitude perturbation. 
A more detailed work will extend these stability analysis to a broader parameter space regime.




% Acknowledgement
\section{ACKNOWLEDGMENTS}
The reference section will follow the ``Acknowledgment'' section.  References should be numbered using Arabic numerals 
followed by a period (.) as shown below, and should follow the format in the below examples.



% References
\nocite{*}
\bibliographystyle{aipnum-cp}%
\bibliography{aip_castro_aceves}%






\end{document}
